\subsection{Motivation \& Background}

\subsubsection{Goals, Bounds, and Impact}

The primary goal of this section of the thesis is demonstration of a novel
time integration scheme driving a canonical or especially demonstrative
reacting flow problem (ideally one that points directly to real-world impact)
with improved efficacy identified by either superior accuracy and stability or
faster performance. Also critical in this effort will be maintaining a solution
that lies on the ideal gas constraint manifold (i.e. a physical solution) - this
has been identified in early work as a potential sticking point when fully
differential approaches to solving the governing equations (as opposed to
differential-algebraic approaches that directly satisfy the constraints) are used.
While the currently working method of solving the equations uses a chemical
kinetic Jacobian modified to include the algebraic constraints explicitly, fully
differential approaches that use attenuation to force the solution towards the
constraint manifold may also be employed to greater effect.

Perhaps the next most prominent goal is a thorough investigation of
the design space presented by the multi-rate framework we will use for time
integration (see \emph{The Plan}). The design choices inherent to this
scheme include:
\begin{itemize}
\item{Evaluation order ("fastest-first" vs. "slowest-first")}
\item{Re-extrapolation}
\item{Inclusion of additional history beyond order requirements (shown in \cite{mikida2019multi} to provide real-axis stability improvement)}
\item{Which solution components to include in error estimation for adaptive timestep
      control}
\item{Whether error control is accomplished through timestep control or step ratio control}
\end{itemize}
Another critical component would be comparison of our scheme with CVODE as
a proxy for the state of the art in time integration of chemically reacting
flows, ideally achieving superiority in terms of observed temporal accuracy
of the entire system, or in terms of performance (most accurately measurable
by a reduction in the number of chemistry right-hand-side evaluations required
to reach a given solution time).

Meanwhile, an aspect of the algorithms being implemented that has been identified
early as beyond our bounds is the presence of multi-variable solves in a number of
potential adaptive implicit integration scenarios, namely:
\begin{itemize}
\item{Coupled implicit right-hand-sides}
\item{Use of adaptive timestep controllers where the local error estimate
      of all solution components makes use of implicit and explicit state
      estimates.}
\end{itemize}

With these goals and bounds in mind, the proposed thesis has the potential to
have a significant impact on the simulation of reacting flows, of which there
are countless real-world applications including (but not limited to) subsonic
and supersonic combustion for propulsion (gas turbines, ramjets/scramjets,
rockets), furnaces and residential heating, and solution of problems relating
to atmospheric sciences.

The proposed work would also serve to augment the capabilities of Leap and Dagrt,
two Python packages for code generation of time integrators, by implementing the
proposed integrators. While this work would focus the attention of those new
integrators on reacting flows, additional useful application of implicit-explicit
multi-rate with error-based step control amongst the software's user base is
possible, if not probable.

Finally, the proposed work would also formally introduce new capabilities
to another existing combustion software package, PyJac-V2. This software is
responsible for code generation of callable source term and Jacobian functions
for chemical kinetics, and our work (given its application to problems that
often have large temperature ranges, not to mention a focus on high accuracy)
would add a NASA9 polynomial representation of thermodynamic quantities to
its codebase.

%===========================================================================
\subsection{Governing Equations}

The problem at hand involves the simulation of chemically reacting flows, the
governing equations of which amount to the Navier-Stokes equations with (in
our form):
\begin{itemize}
\item{A source term added to the energy equation in the form of heat flux}
\item{The addition of a governing equation for the rate of change of the
      mass fractions of each species in the chemical mechanism simulated}
\item{The addition of governing equations specifying the rate of change of
      temperature and pressure, derived from a constant internal energy
      assumption (for the former) and the ideal gas law (for the latter).}
\end{itemize}
These equations are given below, along with a brief explanation of the notation/nomenclature
used.
\begin{align}
\frac{\partial \rho}{\partial t} + \frac{\partial}{\partial x_{j}}(\rho u_{j}) &= 0 \\
\frac{\partial}{\partial t}(\rho u_{i}) + \frac{\partial}{\partial x_{j}}(\rho u_{i} u_{j} + P\delta_{ij} - \tau_{ij}) &= 0 \\
\frac{\partial}{\partial t}(\rho E) + \frac{\partial}{\partial x_{j}}((\rho E + P)u_{j} + q_{j} - u_{i}\tau_{ij}) &= 0 \\
\frac{\partial}{\partial t}(\rho Y_{k}) + \frac{\partial}{\partial x_{j}}(\rho Y_{k} u_{j} - \varphi_{ki}) &= W_{k}\dot{\omega}_{k} \\
\frac{\partial T}{\partial t} + \frac{\sum_{k=1}^{N_{sp}}U_{k}(T)\frac{\partial \rho Y_{k}}{\partial t}\frac{1}{W_{k}}}{\sum_{k=1}^{N_{sp}}[C]_{k}C_{v,k}(T)} &= 0 \\
\frac{\partial P}{\partial t} - \frac{R}{V}(T\frac{\partial n}{\partial t} + \frac{\partial T}{\partial t}n) &= 0
\end{align}
In these equations, $\rho$ is the fluid density, $u_{i}$ is the velocity in the $i$-th direction, and $E$ is the total energy.
$\delta_{ij}$ is the Kronecker delta, $P$ is the pressure, $T$ is the temperature, and $Y_{k}$ is the mass fraction of
species $k$. $[C]_{k}$ is the concentration of species $k$ in the gas mixture, $C_{v,k}(T)$ is the specific heat at constant volume
of species $k$ at temperature $T$, and $U_{k}(T)$ is the internal energy of species $k$ at temperature $T$. These quantities
are defined using the NASA 9-coefficient polynomial parameterization \cite{mcbride2002nasa}, such that
\begin{align}
\frac{C_{v,k}(T)}{R} &= a_{0} + a_{1}T + a_{2}T^{2} + a_{3}T^{3} + a_{4}T^{4} - 1 \\
\frac{U_{k}(T)}{RT} &= a_{0} + \frac{a_{1}}{2}T + \frac{a_{2}}{3}T^{2} + \frac{a_{3}}{4}T^{3} + \frac{a_{4}}{5}T^{4} + \frac{a_{5}}{T} - 1,
\end{align}
where the $a$ coefficients are defined per-species for an arbitrary number of temperature regions.

Returning to the Navier-Stokes equations (1) - (6), $N_{sp}$ is the number of species in the chemical mechanism,
$W_{k}$ is the molecular weight of species $k$, $n$ is the number of moles of the gas mixture, V is the volume
of the gas, and $\dot{\omega}_{k}$ is the net production rate of species $k$. $\tau_{ij}$ is the viscous stress
tensor, $\varphi_{ki}$ are the diffusion fluxes, and $q_{j}$ are the heat fluxes, defined by
\begin{align}
q_{j} = - \frac{\partial (\lambda T)}{\partial x_{j}} + h_{k}\varphi_{kj}.
\end{align}
In this expression, $\lambda$ is the thermal conductivity, and $h_{k}$ is the enthalpy of species $k$.
As for the diffusion fluxes, these are defined using a mixture-averaged approach - that is, $\varphi_{ki}$
is given by
\begin{align}
\varphi_{ki} = \varphi_{ki}^{*} + \varphi_{ki}^{c},
\end{align}
where $\varphi_{ki}^{*}$ is the mixture-averaged approximation, and $\varphi_{ki}^{c}$
is a correction term to ensure mass conservation. The mixture-averaged approximation
is defined by
\begin{align}
\varphi_{ki}^{*} = -\rho D_{k,m}\frac{W_{k}}{W} \frac{\partial X_{k}}{\partial x_{i}}
\end{align}
where $D_{k,m}$ is the mixture-averaged diffusivity of species $k$, $W$ is the mean
molecular weight, and $X_{k}$ is the mole fraction of species $k$. Finally, the
correction term is given by
\begin{align}
\varphi_{ki}^{c} = -Y_{k} \sum_{n=1}^{N_{sp}} \varphi_{ni}^{*}
\end{align}

In this set of governing equations (1) - (6), the mass fraction source term
on the right side of equation (4) is typically observed to be stiff relative
to the surrounding equations governing the motion of the gas mixture and the change
in its physical properties. This presents a time integration problem in that this term
alone typically dictates either an oppressively low timestep or the need for
an implicit approach, often using tools such as CVODE. In practice, fluid solvers
also often decouple an explicit treatment of the Navier-Stokes equations (1) - (3)
from the implicit treatment of the chemical kinetics, resulting in a splitting
approach that is at best first order in time.

The topic of this section of the proposal, then, is to answer the following
research question: \textbf{can we derive performance and/or temporal accuracy benefits
from an application of multi-rate Adams integrators to this set of governing
equations?}

%===========================================================================
\subsection{Numerical Methods}

\subsubsection{Adams Methods}

To fix notation for new schemes developed later, we here give a brief
derivation of a standard Adams-Bashforth (AB) integrator, as described
in~\cite{bashforth1883attempt}.  We start with a model IVP given by
\begin{align}
\frac{\text{d}y}{\text{d}t} = F(t,y), \quad y(0) = y_{0}. \notag
\end{align}
This is the form that results from a method of lines (MOL) approach to solving
time-dependent partial differential equations like those considered in this
study.  We approximate the time dependency of the right-hand side function with
a polynomial with coefficients $\boldsymbol{\alpha}$ (formed by interpolating
past values of $F(t,y)$), extrapolate with that polynomial approximation, and
integrate the extrapolant.  We use a Vandermonde matrix to construct a linear
system to obtain the coefficients $\boldsymbol{\alpha}$ to be used in
extrapolation from history values:
\begin{align}
V^{T} \cdot \boldsymbol{\alpha} = \int_0^{\Delta t} \tau^{i} d\tau = \frac{(\Delta t)^{i}}{i+1}, \quad i = 1,2...,n, \quad V = \begin{bmatrix}
    1 & t_{1} &  \hdots   & t_{1}^{n-1}  \\
    1 & t_{2} & \hdots &  t_{2}^{n-1} \\
      \vdots  & \vdots  &  \ddots   &  \vdots   \\
       1 &   t_{n}  &  \hdots  & t_{n}^{n-1} \\
        \end{bmatrix}, \label{eq:vandermonde}
\end{align}
where $\int_0^{\Delta t} \tau^{i} d\tau$ is a vector evaluating the integral of
the interpolation polynomial, and $V$ is the Vandermonde matrix with monomial
basis and nodes $t_{1}, t_{2}, \hdots t_{n}$, corresponding to past time
values. In \eqref{eq:vandermonde}, $n$ is equal to the order of the integrator,
and $t_{i}$ are the time history values, with $0 \leq t_{1} < t_{2} \hdots <
t_{n}$.  The coefficients $\boldsymbol{\alpha}$ are used to extrapolate to the
next state via
\begin{align}
y(t_{i+1}) = y(t_{i}) +& \alpha_{1}F(t_{i-n},y_{i-n}) + \alpha_{2}F(t_{i-n-1},y_{i-n-1}) \notag \\
+& \cdots + \alpha_{n}F(t_{i},y_{i}). \label{eq:ab_general}
\end{align}
Clearly, the length of the past history needed to calculate a step (and, thus, the memory required) influences the order of accuracy attained.

An alternative time integration method is required for the first few time steps
(the exact number of which is dependent on the number of history values needed)
in order to establish right-hand side history and "bootstrap" the method.  We
use a third-order Runge-Kutta (RK3) integrator~\cite{heun1900neue} to bootstrap
the third-order AB methods, whereas a fourth-order Runge-Kutta (RK4)
integrator~\cite{kutta1901beitrag} is used to bootstrap the fourth-order AB
methods.

%% MODIFY TO ADD ADAMS-MOULTON
%% INCLUDE MENTION OF EXTENDED HISTORY WITH CITATION TO PAPER?

\subsubsection{Multi-rate Adams Methods}

We now describe a multi-rate generalization of the scheme, making use of the
algorithm introduced in~\cite{gear1984multirate}.  We consider the following
model system with ``fast'' and ``slow'' solution components:
\begin{align}
\frac{\textrm{d}}{\textrm{d}t}\left( \begin{array}{c} f(t) \\ s(t) \end{array} \right) = \left( \begin{array}{c} a_{f}(f,s) \\ a_{s}(f,s) \end{array} \right). \label{eq:mrab}
\end{align}
With this in mind, we can set a slow (larger) time step $H$ for $a_{s}$ such
that we maintain stability in the integration of the slow component.  We also
set a fast time step $h$ for $a_{f}$ such that $H$ is an integer multiple of
$h$, and define the ratio between the two, $\text{SR} = H/h$, as the step ratio
of the MRAB scheme. While the results presented here make use of only two 
separate state components, each with its own right-hand side function and
independent rate, the theory is readily extensible to any number of rates.

%In the reacting flow formulation with which we are concerned, we define
%the fast and slow components of our Navier-Stokes solution as the conserved
%variables $Q_{i} = [\rho_{i}, (\rho \vec{u})_{i}, (\rho E)_{i}]^T$ on each
%grid, that is (using a two-grid case as an example): $f = Q_{1}$, $s = Q_{2}$,
%where the subscripts of the vectors $Q$ indicate global grid number.  
%We assume Grid 1 to be the grid with the fast-moving component of the solution, 
%be it due to physical behavior or finer mesh spacing.  Each
%right-hand side function $a_{f}$ and $a_{s}$ is a function of both the slow and
%fast states $s$ and $f$ --- this coupling between the right-hand side functions
%is, in the case of our application of this theory to overset meshes, realized
%by the SAT penalty interpolation discussed in Section~\ref{interp}.

Within this two-component scheme, a few design choices are available:
\begin{itemize}
\item The order in which we evaluate and advance the solution components.
	Namely, two primary options are advancing the fast-evolving solution
	component through all of its micro-timesteps $h$ and waiting to perform
	the single macro-timestep $H$ required for the slow component until the
	end (a ``fastest-first'' scheme, per the nomenclature of
	\cite{gear1984multirate}), or pursuing an algorithm in which the slow
	component is instead advanced first.
\item For slowest-first evaluation schemes, the choice of whether or not to
	re-extrapolate the slow state after additional state and right-hand
	side information is gathered at the micro-timestep level.
\end{itemize}
Empirical observations on the effects of these choices are made
in \cite{klockner2010high}.  It is useful to step through a brief example
of a multi-rate Adams-Bashforth integrator, using a system with a
fast component requiring twice as many timesteps as the slow component
to remain well-resolved ($\text{SR}=2$).  We lay out the steps of a third-order
fastest-first MRAB scheme with no re-extrapolation, assuming that $a_{s}$
evolves at the slow rate (macro-timestep $H = 2h$) and $a_{f}$ evolves at the
fast rate (micro-timestep $h$).  $\hat{a}$ denotes extrapolants of the
right-hand side functions as polynomial functions of both time $t$ and the set
of history values $\vec{a}_{\text{hist}}$: $\hat{a} = P(t,
\vec{a}_{\text{hist}})$.  These polynomials approximating the evolution of
$a_{f}$ and $a_{s}$ in time are what we will integrate to march $a_{f}$ and
$a_{s}$, and will be updated to replace older history values with new
right-hand side evaluations during the course of integration through a
macro-timestep $H$.  We assume availability of right-hand side histories to
start the AB method.
\begin{itemize}[leftmargin=0.5in]
\item[Step 1:] Form the polynomial extrapolants we will integrate, per the AB
	methods described in the previous subsection:	
\begin{align}
\hat{a}_{f}(t) = P\Big(t, [a_{f}(f(t_{i-2}), s(t_{i-2})), a_{f}(f(t_{i-1}), s(t_{i-1})), a_{f}(f(t_{i}), s(t_{i}))]\Big) \notag \\
\hat{a}_{s}(t) = P\Big(t, [a_{s}(f(t_{i-4}), s(t_{i-4})), a_{s}(f(t_{i-2}), s(t_{i-2})), a_{s}(f(t_{i}), s(t_{i}))]\Big) \notag
\end{align}	
	The right-hand side history
	values of $a_{f}$ (used to form $\hat{a}_{f}$) have been obtained at
	time points $t_{i - 2} = t - 2h$, $t_{i-1} = t - h$, and current time
	$t_{i} = t$, whereas the right-hand side history values of $a_{s}$
	(used to form $\hat{a}_{s}$) have been obtained at time points $t_{i-4}
	= t - 2H$, $t_{i-2} = t - H$, and $t_{i} = t$.
\item[Step 2:] March both $f$ and $s$ to time $t_{i+1}$ by integrating the
	polynomial extrapolants $\hat{a}_{s}(t)$ and $\hat{a}_{f}(t)$ formed in
	Step~1:
\begin{align}
f(t_{i+1}) = f(t_{i}) + \int_{t_{i}}^{t_{i+1}} \hat{a}_{f}(\tau) d\tau, \notag \\
s(t_{i+1}) = s(t_{i}) + \int_{t_{i}}^{t_{i+1}} \hat{a}_{s}(\tau) d\tau. \notag
\end{align}
This results in a set of intermediate values $f(t_{i+1})$ and $s(t_{i+1})$.
\item[Step 3:] Evaluate the fast right-hand side $a_{f}(f(t_{i+1}), s(t_{i+1}))$.
\item[Step 4:] Update the set of right-hand side history values for $a_{f}$ to include these new values, and construct a new extrapolant $\hat{a}_{f}$:
\begin{align}
\hat{a}_{f}(t) = P\left(t, [a_{f}(f(t_{i-1}), s(t_{i-1})), a_{f}(f(t_{i}), s(t_{i})), a_{f}(f(t_{i+1}), s(t_{i+1}))]\right). \notag
\end{align}
\item[Step 5:] March $s$ to time $t_{i+2}$ by integrating the extrapolant formed in Step~1:
\begin{align}
s(t_{i+2}) = s(t_{i}) + \int_{t_{i}}^{t_{i+2}} \hat{a}_{s}(\tau) d\tau. \notag
\end{align}
\item[Step 6:] March $f$ to time $t_{i+2}$ by integrating the extrapolant formed in Step~3:
\begin{align}
f(t_{i+2}) =  f(t_{i+1}) + \int_{t_{i+1}}^{t_{i+2}} \hat{a}_{f}(\tau) d\tau. \notag
\end{align}
\item[Step 7:] Evaluate $a_{f}(f(t_{i+2}), s(t_{i+2}))$ and $a_{s}(f(t_{i+2}), s(t_{i+2}))$ and update extrapolants:
 \begin{align}
\hat{a}_{f}(t) = P\Big(t, [a_{f}(f(t_{i}), s(t_{i})), a_{f}(f(t_{i+1}), s(t_{i+1})), a_{f}(f(t_{i+2}), s(t_{i+2}))]\Big), \notag \\
\hat{a}_{s}(t) = P\Big(t, [a_{s}(f(t_{i-2}), s(t_{i-2})), a_{s}(f(t_{i}), s(t_{i})), a_{s}(f(t_{i+2}), s(t_{i+2}))]\Big). \notag
\end{align}
\end{itemize}
The scheme evaluates the fast-evolving right-hand side $a_{f}$ twice per
macro-timestep $H$, whereas the slowly-evolving right-hand side $a_{s}$ is only
evaluated once.  For the results shown later, this is the scheme we will use,
albeit generalized to different step ratios $\text{SR} = H/h$.  

\subsubsection{Timestep Control Algorithm}

Early implicit-explicit runs of small-scale reacting flow problems
quickly identified a need to adapt the timestep of the integrator based
on relative and/or absolute local error demands. We therefore employ a
timestep control algorithm which, in its current form, is that of ODE45, wherein
a timestep multiplier $r$ is calculated based on a local error estimate constructed from
state estimates of two (differing) explicit orders:
\begin{align}
r = \frac{\|s_{q+1} - s_{q}\|_{2}}{\text{ATOL} + \text{RTOL} \cdot \text{max}(\|s_{q}\|_{2}, \|s_{q+1}\|_{2})}
\end{align}
Here, $s_{q}$ is the state estimate obtained using an order-$q$ scheme, and ATOL and RTOL are absolute and
relative error tolerances specified by the user. Once $r$ is calculated, the timestep is \emph{decreased} if
$r>=1$ via
\begin{align}
\Delta t = 0.9\Delta t (r)^{-1/q}
\end{align}
and \emph{increased} if $r<1$ via
\begin{align}
\Delta t = 0.9\Delta t (r)^{-1/(q+1)}
\end{align}

This added capability alone presents a number of questions in terms of how 
best to accomplish this error control. In particular, step ratio adjustment in situ,
rather than timestep adjustment, to meet error needs provides an implementation
challenge in terms of code generation of these integrators, whereas the choice of
which solution component(s) are involved in error estimation/timestep control also
presents a design decision that could have notable effect on numerical performance.

%===========================================================================
\subsection{Validation}

With the plan for the numerical methods to be implemented in place, the dominant
planning question then becomes: what is the test problem to which these new methods will
be applied to establish their viability? The current plan is to use a reacting mixing layer
problem with a hyperbolic tangent velocity profile for the baseflow (as used by \cite{michalke1964inviscid}
and \cite{blumen1970shear}), with the San-Diego 9-species chemical mechanism employed for
the chemical kinetics. By perturbing this baseflow with the mode having the highest growth
rate to instability (based on the inviscid analyses of \cite{michalke1964inviscid} and
\cite{blumen1970shear}), we can establish a fast-evolving solution component (the chemical
kinetics/mass fractions of the species) and a slow-evolving solution component (the fluid),
both possessing accessible means of rate modification (for the chemistry, we can accomplish
this by scaling the pre-exponential factors in the Arrhenius reaction rates, and for the
fluid, we can scale the magnitude of the initial crossflow).

\begin{figure}
\centering
\includegraphics[width=0.3\linewidth,trim=4 4 4 4,clip]{figures/hyperbolic_tangent.png}
\caption{Perturbed vertical momentum of the hyperbolic tangent example - multispecies cold flow.}
\label{fig:hyperbolic_cold_rhov2}
\end{figure}

In doing so, what should result is a problem with ample testbed capabilities for our
new multi-rate integrators, and also a problem firmly rooted in physical utility in that
it provides the simplest model for scramjet/ramjet combustion.

%===========================================================================
\subsection{Results}
(Insert text here)

%===========================================================================
\subsection{Outlook}

\subsubsection{Current Status}

At present, implicit-explicit time integration of reacting flows using Runge-Kutta
based methods in conjunction with code generation of source terms and Jacobians for
the chemical mechanisms is implemented and undergoing testing with the fluid solver
application. Implicit Adams methods (with single-rate Adams-Moulton being the baseline)
for the purposes of chemistry integration are also being implemented, along with
an application of the error-informed timestep control algorithm to both
single-rate and multi-rate Adams methods. Construction of implicit-explicit multi-rate
Adams methods with error-informed adaptivity are well underway, with testing
of these new methods on small-scale Cantera reactor system problems in progress.

As for the reacting crossflow validation problem, an initial setup employing
cold flow (no autoignition) with the perturbed baseflow applied to a nine-species
(San Diego) gas mixture is complete, with validation via the growth rate of the
unstable mode ongoing. Also implemented in the reacting flow solver is a one-dimensional
laminar free flame problem, which may also be used as a first-pass validation (via
comparison to the Cantera-estimated steady-state flame speed) for new integration
methods for reacting flows as they become available.

\subsubsection{Risk Mitigation}

An important question to ask amidst all of these promises is: 
textbf{what if this doesn't work out?} Assuming demonstration of improved
performance or accuracy of the \emph{multi-rate} integrators over the
state-of-the-art is for reasons unforseen impossible, a "fallback" takeaway
would be demonstration of improvement over the CVODE-fluid first-order splitting
approach via Leap implementation of an existing IMEX Runge-Kutta based scheme
that treats the chemistry implicitly via code-generated analytical Jacobians.
Given the use of a coupled implicit-explicit scheme (for which high order has
already been proven in other circumstances), as well as a chemical Jacobian
that is more accurately obtained than via finite differences (the method
of CVODE), this outcome should be attainable at minimum.

%===========================================================================
