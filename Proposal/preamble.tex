\raggedbottom
\renewcommand{\capfont}{\footnotesize}
\renewcommand{\caplabelfont}{\footnotesize\bfseries}
%%----------------------------------------------------------------------------------------------------%%
%%------------------------------------Sprache---------------------------------------------------------%%
\usepackage[T1]{fontenc}
\usepackage[latin1]{inputenc}
\usepackage[english]{babel}
\setlength{\parindent}{0pt}
\usepackage{caption}
%%----------------------------------------------------------------------------------------------------%%
%%-------------------------------------Bilder---------------------------------------------------------%%
%\usepackage[rflt]{floatflt}
\usepackage{epsfig,wrapfig,epstopdf,psfrag}
\usepackage{subcaption}
\usepackage[hang]{footmisc}
\usepackage{import}
\usepackage[maxfloats=25]{morefloats}
%%----------------------------------------------------------------------------------------------------%%
%%------------------------------------Fussnoten-------------------------------------------------------%%
\setlength{\footnotesep}{0.5em} 
\setlength{\footnotemargin}{0.5em} 
%%----------------------------------------------------------------------------------------------------%%
%%-------------------------------------Mathematische Symbole---------------------------%%
\usepackage{pgf,tikz}
\usetikzlibrary{arrows}
\definecolor{cqcqcq}{rgb}{0.7529411764705882,0.7529411764705882,0.7529411764705882}
\usepackage{amsmath,amssymb}
\usepackage{esint}
\usepackage{nicefrac}
\usepackage{pgfplots}
%%----------------------------------------------------------------------------------------------------%%
%%-------------------------------------Programming----------------------------------------------------%%
\usepackage[curves]{struktex}
\usepackage{fancyvrb}
%%----------------------------------------------------------------------------------------------------%%
%%-------------------------------------SI unit/Abkürzungen--------------------------------------------%%
\usepackage{siunitx}
\usepackage{acronym}
%%----------------------------------------------------------------------------------------------------%%
%%-------------------------------------Tabellen-------------------------------------------------------%%
\usepackage{ragged2e,array,longtable,lscape}
\usepackage{multirow}
\usepackage{tabularx}
\usepackage{booktabs}
\usepackage[figuresright]{rotating}
%%----------------------------------------------------------------------------------------------------%%
%%-------------------------------------Code-----------------------------------------------------------%%
\usepackage{scrhack}
\usepackage{listings}
%\usepackage[framed,numbered,autolinebreaks,useliterate]{mcode}
%\lstset{language=tcl,basicstyle=\scriptsize,numbers=left,numberstyle=\scriptsize,stepnumber=10, numberfirstline=true}
\usepackage{printlen}	
\definecolor{mygreen}{RGB}{28,172,0} % color values Red, Green, Blue
\definecolor{mylilas}{RGB}{170,55,241}
\lstset{language=Matlab,%
	basicstyle=\ttfamily\scriptsize,    
    %basicstyle=\color{red},
    breaklines=false,%
    morekeywords={matlab2tikz},
    keywordstyle=\color{blue},%
    morekeywords=[2]{1}, keywordstyle=[2]{\color{black}},
    identifierstyle=\color{black},%
    stringstyle=\color{mylilas},
    commentstyle=\color{mygreen},%
    showstringspaces=false,%without this there will be a symbol in the places where there is a space
    numbers=left,%
    numberstyle={\tiny \color{black}},% size of the numbers
    numbersep=9pt, % this defines how far the numbers are from the text
    emph=[1]{for,end,break},emphstyle=[1]\color{red}, %some words to emphasise
    %emph=[2]{word1,word2}, emphstyle=[2]{style},    
}
%%----------------------------------------------------------------------------------------------------%%
%%-------------------------------------Zitieren-------------------------------------------------------%%
\usepackage{cite}
%%----------------------------------------------------------------------------------------------------%%
%%-------------------------------------PDF-LaTeX------------------------------------------------------%%
\usepackage[colorlinks,
                    pdfpagelabels,
                    pdfstartview = FitH,
                    bookmarksopen = false,
                    bookmarksnumbered = true,
                    hypertexnames = false
                    ]{hyperref}
\hypersetup{
                    pdftoolbar=true,          % show Acrobat’s toolbar?
                    pdfmenubar=true,          % show Acrobat’s menu?
                    pdfauthor={Fabian Dettenrieder},     % author
                    colorlinks=true,          % false: boxed links; true: colored links
                    linkcolor=black,          % color of internal links
                    citecolor=black,          % color of links to bibliography
                    filecolor=black,          % color of file links
                    urlcolor=black            % color of external links
}
\usepackage{pdfpages}
\usepackage[section]{placeins}
%%----------------------------------------------------------------------------------------------------%%
%%-------------------------------------Farbe----------------------------------------------------------%%
\usepackage{color}
\newcommand{\red}[1]{\textcolor{red}{#1}}
\newcommand{\blue}[1]{\textcolor{blue}{#1}}
\newcommand{\green}[1]{\textcolor{green}{#1}}
\newcommand{\white}[1]{\textcolor{white}{#1}}
\definecolor{darkgreen}{rgb}{0,0.5,0}
\definecolor{darkred}{rgb}{0.8,0.1,0}
\definecolor{darkblue}{rgb}{0.1,0.5,0.8}
	\newcommand{\darkred}[1]{\textcolor{darkred}{#1}}
\newcommand{\darkblue}[1]{\textcolor{darkblue}{#1}}
\newcommand{\darkgreen}[1]{\textcolor{darkgreen}{#1}}
%%----------------------------------------------------------------------------------------------------%%
%%-------------------------------------neue Befehle---------------------------------------------------%%
% Automatisches Hinzufügen bei Referenzen
\newcommand{\figref}[1]{(Abb.~\ref{#1})}
\newcommand{\figrefl}[1]{Abbildung~\ref{#1}}
\newcommand{\tbref}[1]{(Tab.~\ref{#1})}
\newcommand{\tbrefl}[1]{Tabelle~\ref{#1}}
\newcommand{\chpref}[1]{\S~\ref{#1}}
\newcommand{\secref}[1]{\S~\ref{#1}}
\newcommand{\appref}[1]{Anhang~\ref{#1}}
\newcommand{\longeqref}[1]{Gleichung~\eqref{#1}}
\newcommand{\eref}[1]{(\ref{#1})}
\newcommand{\mbf}[1]{\mathbf{#1}}
\newcommand{\mrm}[1]{\mathrm{#1}}
\newcommand{\bs}[1]{\boldsymbol{#1}}
\newcommand{\mcl}[1]{\mathcal{#1}}
\newcommand{\abs}[1]{\ensuremath{\left\vert#1\right\vert}}
%%----------------------------------------------------------------------------------------------------%%
%%-------------------------------------Kopfzeilen-----------------------------------------------%%
\usepackage{scrpage2}
\clearscrheadings
\clearscrplain
\ohead[\pagemark]{\pagemark}
\rehead{\leftmark}
\lohead{\rightmark}
\addtocontents{toc}{\protect\markright{}}
\addtocontents{lof}{\protect\markright{}}
\addtocontents{lot}{\protect\markright{}}
\automark[section]{chapter}

%%----------------------------------------------------------------------------------------------------%%
%%-------------------------------------Listen------------------------------------------------------%%
\setlength{\itemsep}{0ex}
\setlength{\parsep}{0ex}
\setlength{\parskip}{2mm}
\setcounter{secnumdepth}{2}
\setcounter{tocdepth}{2}
%%----------------------------------------------------------------------------------------------------%%
%%----------------------------------------------------------------------------------------------------%%