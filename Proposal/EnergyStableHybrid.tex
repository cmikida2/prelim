\subsection{Motivation \& Background} \label{sec:hybrid_goals}

%\subsubsection{Historical Context}

When simulating problems in computational fluid dynamics involving complex
geometries, numerical difficulties are often encountered that require
employing one of a number of different approaches to mitigate loss of accuracy and/or stability
due to poor spatial resolution. While finite-difference approaches to spatial
discretization are popular due to their simplicity and efficiency when working
with smooth solutions, their geometric flexibility is typically limited in spite
of the well-documented ability to use coordinate transforms to create intricate
curvilinear meshes, as well as the ability to use overset meshes \cite{bodony2011provably, noack2005summary}
to produce higher resolution in particularly troublesome sections of the spatial domain.

In particular, the summation-by-parts (SBP) methodology pioneered by Strand \cite{strand1994summation},
which mimics
integration-by-parts in the discrete sense and is provably stable via the energy method when coupled with weak enforcement of boundary conditions (simultaneous approximation term; see \cite{svard2007stable}),
is an attractive method when it comes to finite differences, and its geometric
flexibility has recently been further improved by a large body of work in creating
accurate and stable interfaces between nonconforming blocks. In 2010, Mattsson and
Carpenter \cite{mattsson2010stable} introduced interpolation operators that maintained
the stability and high-order accuracy of the underlying spatial scheme on each block,
but the resulting scheme is based on a fixed refinement ratio and the requirement that
the blocks conform at their corners (the latter of these characteristics was
later removed by Nissen et al. in 2015 \cite{nissen2015stable}). The use of SBP-SAT finite
difference techniques in conjunction with nonconforming grid interfaces has been further
extended to use with the second-order wave equation \cite{wang2016high}, 3D elastic wave
simulations \cite{gao2020energy}, the Schr{\"o}dinger equations \cite{nissen2012stability},
and the advection-diffusion equation \cite{lundquist2018hybrid}. 

Meanwhile, another attractive approach to modeling complex geometries more efficiently
with a localized spatial discretization change is to create a hybrid spatial
discretization that requires an interface between a high-order finite difference
method (used away from the complex boundary) and an unstructured mesh with
improved geometric flexibility (used near the complex boundary). Nordstr{\"o}m and Gong
proposed such a scheme in 2006 \cite{nordstrom2006stable}, coupling a high-order finite difference method
and a second-order finite volume scheme. More recently, in 2016, Kozdon and Wilcox
\cite{kozdon2016stable} developed an interface method between SBP finite-difference
blocks that was based on projection into an intermediate "glue" grid characterized by
piecewise polynomials, with the method also extending to the coupling of SBP
finite-difference methods and discontinuous Galerkin methods. In this work, the
interface method was proven to be stable via the energy method, but the projection operators were
not fully determined and were based in part on an optimization procedure focused on
driving the product of the projection matrices closer to an identity operation.
Furthermore, the method was not extended to three dimensions, and the method is reduced
to first order accuracy if the continuous coordinate transforms on either side of the
interface are dissimilar.
\begin{figure}
\centering
\includegraphics[width=0.7\linewidth,trim=4 4 4 4,clip]{figures/nonconforming_samples.png}
\caption{Two examples demonstrating the use of nonconforming interfaces. Left: Zhu, Chen, Zhong, and Liu \cite{zhu2011hybrid}.
	 Right: Kozdon and Wilcox \cite{kozdon2016stable}}
\label{fig:nonconforming_samples}
\end{figure}
Since, Friedrich et al. \cite{friedrich2018conservative} extended the
method(s) of Kozdon and Wilcox to make the
projection process degree-preserving (a characteristic that had been
previously shown to be problematic by Lundquist and Nordstr{\"o}m \cite{lundquist2016suboptimal}) as well as
energy-stable and conservative, but their approach still results in projection
operators that are not fully determined, relying upon an optimization procedure
to produce the operators that give their results. In 2019, Almquist et al.
\cite{almquist2019order} developed new order-preserving interpolation operators that accomplished
the same goal as Friedrich et al. without increasing the quadrature order, and which
extended to problems with second-order spatial derivatives, but their results again
were confined to problems with a fixed 2:1 refinement ratio. Furthermore, unlike Kozdon
and Wilcox, the methods of both Almquist and Friedrich fail to extend their method
to hybrid structured-unstructured discretizations.

%\subsubsection{Goals, Bounds, and Impact} \label{sec:hybrid_goals}

This section of the thesis targets a lack of fully-determined high-order
accurate and provably stable interface conditions between structured and unstructured
meshes for computational fluid dynamic simulations. The key contribution of this work
will be the construction of an interface between structured (finite difference SBP)
and unstructured (DG) spatial discretizations that does not rely upon
optimization or an unequal number of constraints and unknowns for construction
of the solution projections. Currently, our plan is to match the approach of
\cite{kozdon2016stable}, with the interface being defined by operators $\pmb{P}_{a2b}$
and $\pmb{P}_{b2a}$ that project the solution along the edge of discretization $a$
into discretization $b$, and vice versa.

An additional goal to be pursued in conjunction with a unisolvent procedure while
maintaining accuracy and stability is a resulting projection process that is able to
provide "round-trip" fidelity of projection, meaning that projection from the structured
grid to the unstructured grid and back (and vice versa) results in a solution identical to the starting
point. This characteristic is one that the operators of \cite{kozdon2016stable} lack,
even in projection from one grid into the intermediate glue layer.

The strategy for pursuance of these goals begins with consideration of the
\emph{interior} projection operator only, given the simplicity of the existing
interior stencil relative to the boundary (which features many more unknowns).
Noting also that the compatibility condition, which we will later find to be the
critical condition for proving a stable interface, disallows us from obtaining
exact "round-trip" projection, other conditions may be considered based
on further manipulation or restructuring of the edge energy dissipation relations,
which will be given for both structured and unstructured spatial discretizations
in subsequent sections. Should a procedure for fully-determined interior interface
conditions be applied successfully to a periodic problem, the boundary interface
will subsequently be considered.

%In particular, our aim
%is to create a provably stable, accurate, and fully-determined interface between
%summation-by-parts finite difference discretizations (structured) and discontinuous
%Galerkin methods (unstructured) that doesn't rely on optimization to establish its
%projection operators. Achieving this would provide a more robust option for creating
%localized areas of unstructured meshing (with superior flexibility for discretizing
%complex boundaries/geometries) in existing simulations, providing an alternative to
%comparatively less flexible overset curvilinear meshes or multiblock
%conforming/nonconforming finite difference schemes.

%With all of this borne in mind, our primary goal for the interface method, then,
%is to obtain projection operators that maintain the provable energy stability
%of Kozdon and Wilcox, which are also
%\begin{itemize}
%\item{obtained from a construction procedure that is demonstrably \emph{unisolvent},
%      with an equal number of constraints and unknowns resulting in a square system
%      of equations}
%\item{therefore not based on optimization}
%\item{able to provide "round-trip" fidelity of projection, meaning that projection
%      from the grid to the glue, and then back to the grid, results in a solution
%      identical to the starting point.}
%\end{itemize}

%===========================================================================
\subsection{Governing Equations}

The initial problem to which our new interface method will be targeted is
the acoustic wave equation in two dimensions, in first-order form:
\begin{align}
	&\rho \frac{\partial v_{i}}{\partial t} + \frac{\partial p}{\partial x_{i}} = 0 \hspace{2mm} (i = 1, 2), \label{eq:wave1} \\
	&\frac{\partial p}{\partial t} + \lambda \left(\frac{\partial v_{1}}{\partial x_{1}} + \frac{\partial v_{2}}{\partial x_{2}} \right) = 0 \label{eq:wave2}
\end{align}

As in \cite{kozdon2016stable}, our aim will be to prove the semi-discrete hybrid discretization
stable via the energy method by proving that
\begin{align}
\frac{d\mcl{E}}{dt} \leq 0
\end{align}
where the total energy of the semi-discrete system $\mcl{E}$ is obtained by summing the
energy $E$ of each discretization block:
\begin{align}
\mcl{E} = \sum_{blocks} E.
\end{align}
The definitions of the energy dissipation rates for each
block (each having its own spatial discretization), are given in subsequent
sections.

%===========================================================================
\subsection{Numerical Methods}

\subsubsection{Summation-by-Parts Operators}

For the structured discretization that makes up one component of the proposed hybrid setup, we make use of several
finite difference operators that possess the summation-by-parts (SBP) property. Taking two matrices
${H,Q}$, we here state that these two matrices are SBP matrices of order $p$ provided 
\begin{itemize}
\item $H^{-1}Q v$ is an order $h^{p}$ approximation to $\partial/\partial x$, where $h$ is the spatial step size in one dimension.
\item $H$ is a symmetric positive-definite matrix.
\item $Q + Q^{T} = \text{diag}(-1,0,0,...,0,1)$.
\end{itemize}
These conditions together ensure that the discrete version of the integration by parts property holds; that is,
\[\langle H^{-1}Q x , y\rangle_{H} = x_{N} y_{N} -  x_{1} y_{1} - \langle x,H^{-1}Q y\rangle _{H}. \]

%%%%% CANDIDATE FOR REMOVAL
%The resulting operators can be either explicit (in this case, $H$ is purely diagonal) or implicit.
%This theory, originally presented in \cite{strand1994summation}, can also be extended to higher
%dimensions using Kronecker products.  For example, in two dimensions, the matrices $P^{-1} G_x$
%and $P^{-1} G_y$ define the $x$- and $y$- derivatives on a two-dimensional grid:
%\begin{align}
%P = H_{x} \otimes H_{y} \notag \\
%G_{x} = Q_{x} \otimes H_{y} \notag \\
%G_{y} = H_{x} \otimes Q_{y} \notag
%\end{align}
%Note that this formulation assumes that we have ${H_{x},Q_{x}}$, a pair of $n_{x} \times n_{x}$ SBP
%matrices of approximation order $p$, and ${H_{y},Q_{y}}$, a pair of $n_{y} \times n_{y}$ SBP matrices
%of approximation order $q$. Finally, we also note that these SBP operators do not guarantee strict
%%%%% END CANDIDATE FOR REMOVAL
Note that these SBP operators do not alone guarantee strict
stability for an initial boundary value problem. We must also apply the boundary conditions using a
formulation that permits an energy estimate. More information on these boundary conditions, called
simultaneous approximation term (SAT) boundary conditions, can be found in \cite{svard2007stable,
svard2008stable, bodony2010accuracy}. Additionally, more information on SBP operators
of various orders, including the coefficients themselves, can be found
in \cite{strand1994summation, carpenter1993time, mattsson2004stable}. For this work,
we will employ diagonal-norm SBP operators for the finite difference discretizations.

The SBP-SAT spatial discretization of the governing equations in the previous section
(Equations \eref{eq:wave1} - \eref{eq:wave2}), in two dimensions and including the
effect of coordinate transform from the physical space to a reference space
$[-1,1]\times[-1,1]$, is given by
%\pagebreak
\begin{align}
	&\rho \pmb{J} \frac{d \pmb{v}_{i}}{dt}
	+ \pmb{D}_{1} \pmb{J} \frac{ \pmb{\partial \xi_{1}}}{\pmb{\partial x_{i}}} \pmb{p}
	+ \pmb{D}_{2} \pmb{J}\frac{\pmb{\partial \xi_{2}}}{\pmb{\partial x_{i}}} \pmb{p} =
	- \pmb{{H}}^{-1}\pmb{\mcl{F}}_{v_{i}}, \quad i=1,2,\\
	&\pmb{J}\frac{\partial \pmb{p}}{\partial t} +\lambda\left(
	\pmb{J}\frac{\pmb{\partial \xi_{1}}}{\pmb{\partial x_{1}}} \pmb{D}_{1} v_{1}
	+ \pmb{J}\frac{\pmb{\partial \xi_{2}}}{\pmb{\partial x_{1}}} \pmb{D}_{2} v_{1}
	+ \pmb{J}\frac{\pmb{\partial \xi_{1}}}{\pmb{\partial x_{2}}} \pmb{D}_{1} v_{2}
	+ \pmb{J}\frac{\pmb{\partial \xi_{2}}}{\pmb{\partial x_{2}}} \pmb{D}_{2} v_{2}
  \right)
  =
	-\lambda \pmb{H}^{-1} \pmb{\mcl{F}}_{p},
\end{align}
where bold letters denote matrices or vectors. Here, we have defined the matrices
\begin{align}
	\pmb{H}     =\;& \pmb{H}_{N_{1}}\otimes \pmb{H}_{N_{2}},&
	\pmb{D}_{1} =\;& \pmb{D}_{N_{1}}\otimes \pmb{I}_{N_{2}},&
	\pmb{D}_{2} =\;& \pmb{I}_{N_{1}}\otimes \pmb{D}_{N_{2}},&
\end{align}
where $\pmb{D}_{i} = \pmb{H}_{i}^{-1}\pmb{Q}_{i}$ defines an SBP approximation of $\frac{\partial}{\partial x_{i}}$.
We define corresponding solution vectors as, for example,
\begin{align}
  \pmb{p} &=
  \begin{bmatrix}
    p_{00} & p_{01} & \cdots & p_{0N_{2}} & p_{10} & \cdots & p_{N_{1}N_{2}}
  \end{bmatrix}^{T}
\end{align}
for a domain of size $(N_{1}+1) \times (N_{2}+1)$.
Note also that in these equations, $x_{i}$ refer to coordinates in the physical domain,
whereas $\xi_{i}$ refer to coordinates in the reference domain. The matrix $\pmb{J}$ is
a diagonal matrix
\begin{align}
  \pmb{J} = \text{diag}
  \begin{bmatrix}
	  J_{00} & J_{01} & \hdots & J_{0N_{2}} & J_{10} & \hdots & J_{N_{1}N_{2}} \label{eq:sbp_jac}
  \end{bmatrix},
\end{align}
where the elements $J_{kl}$ correspond to Jacobian determinants at grid point $(k,l)$:
\begin{align}
J = \frac{\partial x_{1}}{\partial \xi_{1}} \frac{\partial x_{2}}{\partial \xi_{2}} -
    \frac{\partial x_{2}}{\partial \xi_{1}} \frac{\partial x_{1}}{\partial \xi_{2}}.
\end{align}
The bolded coordinate transform matrices $\frac{\pmb{\partial \xi_{i}}}{\pmb{\partial x_{j}}}$
and $\frac{\pmb{\partial x_{i}}}{\pmb{\partial \xi_{j}}}$ are defined analogously.

The penalty terms in the above equations implement the boundary conditions via the
SAT method, and are defined as follows:
\begin{align}
  \pmb{\mcl{F}}_{v_{i}} =\;&
	\left(  \pmb{e}_{W} \otimes \pmb{\mcl{F}}^{W}_{v_{i}} \right)
	+ \left(  \pmb{e}_{E} \otimes \pmb{\mcl{F}}^{E}_{v_{i}} \right)
	+ \left( \pmb{\mcl{F}}^{S}_{v_{i}} \otimes \pmb{e}_{S} \right)
	+ \left( \pmb{\mcl{F}}^{N}_{v_{i}} \otimes \pmb{e}_{N} \right),\\
  \pmb{\mcl{F}}_{p} =\;&
    \left( \pmb{e}_{W} \otimes \pmb{\mcl{F}}^{W}_{p} \right)
	+ \left( \pmb{e}_{E} \otimes \pmb{\mcl{F}}^{E}_{p} \right)
	+ \left( \pmb{\mcl{F}}^{S}_{p} \otimes \pmb{e}_{S} \right)
	+ \left( \pmb{\mcl{F}}^{N}_{p} \otimes \pmb{e}_{N} \right).
\end{align}
In these expressions, the subscripts $E$, $W$, $N$, $S$ refer to the edges
of the domain, with the $\pmb{e}$ vectors being equal to zero in every entry except
for the first or last. The actual penalty vectors $\pmb{\mathcal{F}}^{W}_{v_{i}}$ and
$\pmb{\mathcal{F}}^{W}_{p}$ are expressed as 
\begin{alignat}{2}
	\pmb{\mathcal{F}}^{W}_{v_{i}} =\;& \pmb{H}_{2} \pmb{S}_{JW} \pmb{n}^{W}_{i}
	\left( \pmb{p}^*_{W} - \pmb{p}_{W} \right),\qquad&
			  \pmb{\mathcal{F}}^{W}_{p} =\;& \pmb{H}_{2}\pmb{S}_{JW}
  \lambda
	\left( \pmb{v}^*_{W} - \pmb{v}_{W} \right),
\end{alignat}
where the surface normals and Jacobians are given by
\begin{alignat}{3}
	S_{J} =\;& J\sqrt{{\left(\frac{\partial \xi_{i}}{\partial x_{2}}\right)}^2 + {\left(\frac{\partial \xi_{i}}{\partial x_{1}}\right)}^2},\quad&
	n_{1} =\;& \pm\frac{J}{S_{J}} \frac{\partial \xi_{i}}{\partial x_{2}},\quad&
	n_{2} =\;& \pm\frac{J}{S_{J}} \frac{\partial \xi_{i}}{\partial x_{1}}, \label{eq:norms_jac}
\end{alignat}
and the corresponding bolded matrices are constructed to be diagonal and contain
values of these quantities at each grid point, as with $\pmb{J}$ (see Eqn. \eref{eq:sbp_jac}).
The starred vectors $\pmb{p}^{*}$ and $\pmb{v}^{*}$ are specified by the boundary
or interface condition.

It can be shown (\cite{kozdon2016stable}, Appendix A) that the energy dissipation rate
of an SBP block in this case is given by
\begin{align}
  \frac{dE}{dt} = & \sum_{K = \{W,E,S,N\}} \mathcal{D}_{K},\\
  \mathcal{D}_{K} = &
  - \pmb{v}_{K}^{T}\pmb{H}_{K}\pmb{S}_{JK}\pmb{p}^{*}_{K}
  + \pmb{v}_{K}^{T}\pmb{H}_{K}\pmb{S}_{JK}\pmb{p}_{K}
	- {\left(\pmb{v}^{*}_{K}\right)}^{T}\pmb{H}_{K}\pmb{S}_{JK}\pmb{p}_{K}. \label{eq:sbp_diss}
\end{align}
Naturally, this dissipation rate, in conjunction with a definition of
$\pmb{p}^{*}$ and $\pmb{v}^{*}$ based on the numerical characteristics
of the interface, will be critical in proving stability via the energy
method.

\subsubsection{Discontinuous Galerkin Method}

%%%%% CANDIDATE FOR REMOVAL
%The discontinuous Galerkin method we will target our interface method(s) to
%will be briefly summarized here. As in Kozdon and Wilcox, we use a curvilinear
%triangle-based DG method, and we denote a DG element $\Omega_e$ with triangular
%reference element $\tilde{\Omega}$. To obtain a version of the governing equations
%\eref{eq:wave1} - \eref{eq:wave2} usable by DG, we must first obtain the variational
%form by introducing test functions $w_{i}$ and $\varphi$ for the velocities and pressure
%respectively, multiplying them into their respective equations, and integrating over
%the reference element:
%\begin{align}
%  \notag
%  &\int_{\tilde{\Omega}} w_{i}\left[\rho J \frac{\partial v_{i}}{\partial t}
%  + \frac{\partial}{\partial \xi_{1}}\left(J\frac{\partial \xi_{1}}{\partial x_{i}} p\right)
%  + \frac{\partial}{\partial \xi_{2}}\left(J\frac{\partial \xi_{2}}{\partial x_{i}} p\right)\right]\;dA\\
%  \label{eq:wave1_var}
%  &\qquad=
%  -\int_{\partial \tilde{\Omega}} w_{i} S_{J} n_{i} \left( p^* - p\right)ds,
%  \quad i=1,2,\\
%  \notag
%  &\int_{\tilde{\Omega}}
%  \varphi \left[J\frac{\partial p}{\partial t} + \lambda\left(
%    J\frac{\partial \xi_{1}}{\partial x_{1}} \frac{\partial v_{1}}{\partial \xi_{1}}
%  + J\frac{\partial \xi_{2}}{\partial x_{1}} \frac{\partial v_{1}}{\partial \xi_{2}}
%  + J\frac{\partial \xi_{1}}{\partial x_{2}} \frac{\partial v_{2}}{\partial \xi_{1}}
%  + J\frac{\partial \xi_{2}}{\partial x_{2}} \frac{\partial v_{2}}{\partial \xi_{2}}\right)\right]\;dA\\
%  \label{eq:wave2_var}
%  &\qquad=
%  -\int_{\partial \tilde{\Omega}} \varphi\lambda S_{J}
%  \left(v^*-v\right) ds,
%\end{align}
%In these equations, the normals $n_{i}$ and surface Jacobians $S_{J}$ are the same
%as the ones defined by equations \eref{eq:norms_jac}, and the starred
%quantities again refer to boundary and interface conditions implemented via penalization,
%in this instance better referred to as numerical fluxes.
%\begin{align}
%&\int_{\tilde{\Omega}} \left[ w_{i}\rho J \frac{\partial v_{i}}{\partial t}
%- \frac{\partial w_{i}}{\partial \xi_{1}}J\frac{\partial \xi_{1}}{\partial x_{i}} p
%- \frac{\partial w_{i}}{\partial \xi_{2}}J\frac{\partial \xi_{2}}{\partial x_{i}} p\right]\;dA
%=
%-\int_{\partial \tilde{\Omega}} w_{i} S_{J} n_{i} p^*ds.
%\end{align}
%Applying integration by parts to \eref{eq:wave1_var} to move the spatial derivatives to
%the test functions gives us the skew-symmetric form of the variational equations, which
%can then be spatially discretized with the DG method (see Hesthaven and Warburton for more
%detail on this) to obtain:
%%%%% END CANDIDATE FOR REMOVAL
The discontinuous Galerkin (DG) method we will target our interface method(s) to
will be briefly summarized here. As in Kozdon and Wilcox, we use a curvilinear
triangle-based DG method, and we denote a DG element $\Omega_e$ with triangular
reference element $\tilde{\Omega}$. To obtain a version of the governing equations
\eref{eq:wave1} - \eref{eq:wave2} usable by DG, we must first obtain the variational
form by introducing test functions $w_{i}$ and $\varphi$ for the velocities and pressure
respectively, multiplying them into their respective equations, and integrating over
the reference element. Applying integration by parts to the resulting equations to move
the spatial derivatives to the test functions gives us the skew-symmetric form of the
variational equations, which can then be spatially discretized with the DG method 
(see Hesthaven and Warburton \cite{hesthaven2007nodal} for more detail on this) to obtain:
\begin{align}
  \rho \pmb{M}_{J} \frac{d \pmb{v}_{i}}{dt} =&
  \pmb{D}_{1}^{T} \pmb{M}_{1i} \pmb{p}
  + \pmb{D}_{2}^{T} \pmb{M}_{2i} \pmb{p}
  - \sum_{K=1}^{3} \pmb{L}_{K}^{T} \pmb{P}_{bc}^{T} \pmb{n}_{iK}
  \pmb{\Omega}_{bc} \pmb{S}_{JK} \pmb{p}_{K}^{*},\\
  \notag
  \pmb{M}_{J} \frac{d\pmb{p}}{dt} =&
  -\lambda\left(
  \pmb{M}_{11} \pmb{D}_{1} \pmb{v}_{1}
    +
    \pmb{M}_{21} \pmb{D}_{2} \pmb{v}_{1}
    +
    \pmb{M}_{12} \pmb{D}_{1} \pmb{v}_{2}
    +
    \pmb{M}_{22} \pmb{D}_{2} \pmb{v}_{2}
  \right)\\
  &- \sum_{K=1}^{3}\lambda \pmb{L}_{K}^{T} \pmb{P}_{bc}^{T} \pmb{\Omega}_{bc}
  \pmb{S}_{JK} \left(\pmb{v}_{K}^* - \pmb{v}_{K}^{-}\right).
\end{align}
These equations provide a semi-discretization on each element.
%%%%% replacement
The normals $n_{i}$ and surface Jacobians $S_{J}$ are the same
as the ones defined by Eqn. \eref{eq:norms_jac}, and the starred
quantities again refer to boundary and interface conditions implemented via penalization,
in this instance better referred to as numerical fluxes.
%%%%% end replacement
The vector $\pmb{v}_{K}^{-}$ is the normal component of velocity along
edge $K$ of the element evaluated at the cubature points:
\begin{align}
  \pmb{v}^{-}_{K} = \pmb{n}_{1K} \pmb{P}_{bc} \pmb{L}_{K} \pmb{v}_{1}
  + \pmb{n}_{2K} \pmb{P}_{bc} \pmb{L}_{K} \pmb{v}_{2}.
\end{align}
Edge-projected pressures are similarly defined as
\begin{align}
  \pmb{p}_{K}^{-} = \pmb{P}_{bc}\pmb{L}_{K} \pmb{p}.
\end{align}
In these expressions, $\pmb{L}_{K}$ and $\pmb{L}_{K}^{T}$ are operators
which take volume terms to edge $K$ of the element and edge $K$ terms to
the volume respectively. $\pmb{D}_{1}$ and $\pmb{D}_{2}$ are the reference
element differentiation matrices for the two reference coordinate directions.
The projection matrices $\pmb{P}_{c}$ and $\pmb{P}_{bc}$ project from the volume and edge
approximations to the volume and edge cubature points, respectively. The matrices
$\pmb{\Omega}_{c}$ and $\pmb{\Omega}_{bc}$ are diagonal matrices of the integration weights for
the volume and an edge, respectively, defined at the cubature locations. A critical
condition assumed by \cite{kozdon2016stable} for stability of their method is that
$\pmb{\Omega}_{c}$ and $\pmb{\Omega}_{bc}$ are positive definite. Finally, the
element mass matrices in the discretization are defined as
\begin{alignat}{2}
  \pmb{M}_{J}  &= \pmb{P}_{c}^{T} \pmb{\Omega}_{c} \pmb{J} P_{c},\quad&
  \pmb{M}_{ij} &= \pmb{P}_{c}^{T} \pmb{\Omega}_{c} \pmb{J}
		  \frac{\pmb{\partial \xi_{i}}}{\pmb{\partial x_{j}}} \pmb{P}_{c}.
\end{alignat}
Here the diagonal matrices $\pmb{J}$ and $\pmb{\frac{\partial \xi_{i}}{\partial x_{j}}}$
are defined identically to Eqn. \eref{eq:sbp_jac} and defined at the cubature points.
The diagonal matrices $\pmb{S}_{JK}$ and $\pmb{n}_{iK}$ are defined in the same way at the cubature
points.

%Defining the energy in a DG element as
%\begin{align}
%  E =
%  \frac{\rho}{2}v_{1}^{T}M_{J}v_{1}
%  + \frac{\rho}{2}v_{2}^{T}M_{J}v_{2}
%  + \frac{1}{2\lambda}p^{T}M_{J}p
%\end{align}
As with SBP, it can be shown that a single DG element has
the energy dissipation rate
\begin{align}
  \frac{dE}{dt} = & \sum_{K=1}^{3} \mcl{D}_{K},\\
  \mcl{D}_{K} = &
  - {\left(\pmb{v}_{K}^{-}\right)}^{T}\pmb{\Omega}_{bc}\pmb{S}_{JK}\pmb{p}_{K}^{*}
  - {\left(\pmb{p}_{K}^{-}\right)}^{T}\pmb{\Omega}_{bc}\pmb{S}_{JK}
  \left(\pmb{v}^{*}_{K} - \pmb{v}^{-}_{K}\right). \label{eq:dg_diss}
\end{align}
As in the case of SBP, this relation is naturally critical in establishing the
energy stability of the interface, especially considering its direct incorporation
of starred (interface-driven) quantities.

\subsubsection{Interface Method} \label{sec:interface}

Clearly vital in the proof of energy stability for the marriage of the
above methods is the definition of the terms $\pmb{p}^*$ and $\pmb{v}^*$, most
often referred to as the numerical fluxes (in the context of DG) or penalty terms
(in the context of SBP-SAT). At the interface between the two discretizations, these
quantities are naturally contingent on the nature of the interface method, which
for the purpose of this discussion take a basic form based on projection operators
that move the Grid $a$ solutions to Grid $b$,
\begin{align}
\pmb{f}_{a} = \pmb{P}_{b2a}\pmb{f}_{b} \\
\pmb{f}_{b} = \pmb{P}_{a2b}\pmb{f}_{a},
\end{align}
where $f_{a}$ is of course a function defined on Grid $a$ and $f_{b}$ is a function
defined on Grid $b$. Note that it is nowhere stated (and in fact proves to be untrue
for SBP-to-SBP projection using Kozdon and Wilcox's method) that $\pmb{P}_{a2b}\pmb{P}_{b2a} = \pmb{I}$.

While the somewhat unique definitions of $\pmb{p}^{*}$ and
$\pmb{v}^{*}$, which incorporate these projected quantities directly, are perhaps
equally important to provable stability, for now we will focus on the projection
operation itself, using the work of Kozdon and Wilcox as a starting point.
As detailed in their paper, the operators of Kozdon and Wilcox rely primarily
upon the principle of projecting solutions from either discretization
first into an intermediate "glue" space of piecewise continuous polynomials
before projecting the solution onto the other nonconforming discretization. Whereas
the solution on the respective structured and unstructured grids is represented at
the block boundaries on the grid points, the solution on the intermediate glue grid
is represented on each interval by modal coefficient values multiplying Legendre
polynomials defined on the interval $[-1,1]$.

The critical condition that allows proof of stability is referred to as the
\emph{compatibility condition}, and amounts to a simple relation between the
grid-to-glue and glue-to-grid projection operators, the mass matrix of the
piecewise polynomial description used in the glue layer, and a norm operator
$\pmb{H}$:
\begin{align}
\pmb{P}_{f2g}\pmb{M} = \pmb{P}_{g2f}^{T}\pmb{H} \label{eq:compat}
\end{align}
In the case of SBP finite difference discretizations, the norm $\pmb{H}$ is equivalent
to the SBP-norm, which, for the purpose of this work, we will take to be
a diagonal matrix. In the case of discontinuous Galerkin, the $\pmb{H}$ operator is
the diagonal matrix of integration weights at the cubature points at a single
DG element edge, $\pmb{\Omega}_{bc}$.  This compatibility condition, used in conjunction with
the per-block semi-discrete expressions for energy dissipation (Eqn. \eref{eq:sbp_diss} and
\eref{eq:dg_diss}), is what allows for energy stability to be proven.

Also critical in the projection from one spatial discretization to another via this
method is the projection between glue spaces (see Figure \ref{fig:glue_layers}). A
compatibility condition similar to Eqn. \eref{eq:compat} is assumed for this projection,
\begin{align}
	\pmb{M}_{a}\pmb{P}_{gb2ga} = \pmb{P}_{ga2gb}^{T}\pmb{M}_{b}, \label{eq:compat_glues}
\end{align}
but it is shown in \cite{kozdon2016stable} (see Lemma 2.3) that the operators $\pmb{P}_{g2f}$ and
$\pmb{P}_{f2g}$ can also be generalized as projection operators to and from the finest
glue layer.
\begin{figure}
\centering
\includegraphics[width=0.8\linewidth,trim=4 4 4 4,clip]{figures/glue_layers.png}
\caption{Image from Kozdon and Wilcox \cite{kozdon2016stable} demonstrating the glue
         layers present in grid-to-grid projection at a nonconforming interface.}
\label{fig:glue_layers}
\end{figure}
Given its clear importance, the compatibility condition (Eqn. \eref{eq:compat}) is used as an explicit
constraint when constructing the projection operators $\pmb{P}_{f2g}$ and $\pmb{P}_{g2f}$
themselves. The construction, which is laid out in further detail in \cite{kozdon2016stable} (Appendix B),
defines interior and boundary accuracy conditions when projecting from grid to glue
(and vice versa) that match those of corresponding SBP diagonal-norm operators of the
same order. A pictorial representation of this process is given by Figure \ref{fig:projection_layout}.
\begin{figure}
\centering
\includegraphics[width=0.9\linewidth,trim=4 4 4 4,clip]{figures/projection_layout.png}
\caption{Images demonstrating the interior constraints for the projection mimicking SBP4-2
	 operators. Left: projection of modal coefficients of the piecewise polynomial solution
	 on the four surrounding glue grid intervals to the central grid point in the stencil must
	 result in a correct Legendre polynomial solution at point $k$ up to the interior order.
	 Right: projection of the solutions at the four points surrounding interval $k$ to the
	 modal coefficients on interval $k$ must produce the correct piecewise approximation of
	 Legendre polynomials up to interior order on that interval.}
\label{fig:projection_layout}
\end{figure}
These conditions, in conjunction with symmetry conditions, result in a \emph{non-square}
system featuring more unknowns than constraints, in particular near the boundary stencils.
Given the underdetermined nature of the system, the authors then use an optimization
procedure aimed at minimizing the distance between the nearest eigenvalues of $\pmb{B} =
\pmb{P}_{f2g}\pmb{P}_{g2f}$ to produce the final projection operators for a given order.
As mentioned in Section \ref{sec:hybrid_goals}, our goal is to eliminate this optimization
procedure, either by imposing more constraints (such as steps towards round-trip projection
fidelity) or removing extraneous unknowns.

%===========================================================================
\subsection{Validation Plan}

The test problem for validation will match that of \cite{kozdon2016stable},
modeling the two-dimensional acoustic wave equation in first-order form with
the following initial condition:
\begin{align}
  p(x_{1},x_{2},0) &=
  \cos\left(k_{1} x_{1}\right)\cos\left(k_{1} x_{2}\right)
  +
  \sin\left(k_{2} x_{1}\right)\sin\left(k_{2} x_{2}\right),\\
  v_{i}(x_{1},x_{2},0) &= 0,~ \quad i=1,2,
\end{align}
where $k_{1} = \pi/2$ and $k_{2} = \pi$. All exterior boundary conditions are
zero pressure (free-surface) conditions. The exact solution of this problem is
\begin{align}
  p(x_{1},x_{2},t) &=
  \cos\left(\omega_{1} t\right)
  \cos\left(k_{1} x_{1}\right)\cos\left(k_{1} x_{2}\right)
  +
  \cos\left(\omega_{2} t\right)
  \sin\left(k_{2} x_{1}\right)\sin\left(k_{2} x_{2}\right),\\
  v_{1}(x_{1},x_{2},t) &=
  \frac{k_{1}}{\omega_{1}} \sin\left(\omega_{1} t\right)
  \sin\left(k_{1} x_{1}\right)\cos\left(k_{1} x_{2}\right)
  -
  \frac{k_{2}}{\omega_{2}} \sin\left(\omega_{2} t\right)
  \cos\left(k_{2} x_{1}\right)\sin\left(k_{2} x_{2}\right),\\
  v_{2}(x_{1},x_{2},t) &=
  \frac{k_{1}}{\omega_{1}} \sin\left(\omega_{1} t\right)
  \cos\left(k_{1} x_{1}\right)\sin\left(k_{1} x_{2}\right)
  -
  \frac{k_{2}}{\omega_{2}} \sin\left(\omega_{2} t\right)
  \sin\left(k_{2} x_{1}\right)\cos\left(k_{2} x_{2}\right),
\end{align}
where $\omega_{j} = k_{j}\sqrt{2}$ for $j=1,2$. 

As for the spatial discretization, we will employ a square spatial domain
with the left half using diagonal-norm SBP operators for finite differencing,
and the right half using a discontinuous Galerkin method (see Figure \ref{fig:split_domain}). The interior order
of the SBP method, as well as the order of the projection across the interface,
is specified to match the order of the discontinous Galerkin method.
\begin{figure}
\centering
\includegraphics[width=0.9\linewidth,trim=4 4 4 4,clip]{figures/split_domain_wave_double.png}
\caption{Pressure initial condition for test problem, also showing split-domain
	 discretization. The right figure coarsens the wireframe to better show the
	 individual triangular DG elements.}
\label{fig:split_domain}
\end{figure}
Our success metrics for this test problem include:
\begin{itemize}
\item{Demonstrating accurate recreation of the total energy of the system, as observed in the
      numerical simulation of the problem, via a comparison of the total energy calculated from the
      time-integrated state and the total energy calculated via time-integration of a dissipation rate
      derived from the spatial and interfacial schema.}
\item{Demonstrating that this energy does not increase (condition for semi-discrete stability).}
\item{Demonstrating the prescribed order of accuracy using L2 norms.}
\end{itemize}
Upon fulfillment of these goals, extension to three dimensions will be considered.

%===========================================================================
\subsection{Outlook}

\subsubsection{Current Status}

Current work on this project has thus far amounted to reconstruction of the
operators of Kozdon and Wilcox, application of these operators to simple, periodic
SBP-to-SBP advection examples, and verification of their accuracy and energy-stability
properties when focusing on the use of the interior projection stencil (see Figure \ref{fig:advection_projection}).
Having both verified and better understood the construction and capabilities of these operators,
the question then becomes improving upon them, and the chief task to that end is
better understanding the importance of the compatibility condition in both the
construction of the operators and in the proof of energy stability. Namely,
a question being asked via the use of symbolic evaluation tools such as Maxima \cite{li2008maxima}
is whether there are alternative conditions on the grid-to-glue projection that
will provide similar characteristics.
\begin{figure}
\centering
\includegraphics[width=0.9\linewidth,trim=4 4 4 4,clip]{figures/advection_projection.png}
\caption{Application of the glue layer projection theory of \cite{kozdon2016stable} to a
	 scalar advection example with an advection speed of 1. The purple line in the figure
	 on the left is a glue layer to which we project the solution on the rightmost grid points
	 at the periodic boundary in $x$. We then project from the glue layer to the leftmost set
	 of grid points. The plots on the right show both an overall decrease in energy (energy stability)
	 and a good agreement between the energy calculated from the time-integrated state and the energy
	 calculated from time-integrating the energy dissipation rate.}
\label{fig:advection_projection}
\end{figure}
In particular, one question that demands answer pertains directly to the
\emph{interior} operator construction, which in the grid-to-glue direction
amounts to an equal number of degrees of freedom and accuracy constraints
imposed upon the glue grid intervals (see Figure \ref{fig:projection_layout}). In
this case, the projection in the glue-to-grid direction can then be fully
determined by the compatibility condition (Eqn. \eref{eq:compat}), but
it is not clear why this also results in a glue-to-grid projection operator
that fulfills its own accuracy constraints imposed on the grid points.

Additionally, and related to these questions, another avenue of investigation
is in the choices of the degree-of-freedom pattern in the projection from
grid to glue layer and vice versa, as well as whether or not there are other
polynomial bases for the glue layer that maintain positive-definite mass
matrices while also introducing improved accuracy. Finally, the compatibility
condition itself, and namely whether or not there are other conditions on the
projection operators that may fulfill similar goals, may be called into question.

\subsubsection{Risk Mitigation}

Should our goal of a new and unisolvent stable interface method not come to
fruition, a "fallback" goal of sorts is identified in that Kozdon and Wilcox's
method has yet to be extended to three dimensions. This would likely present more of a
technical challenge than an theoretical one, but a study of how the method
responds to certain geometric scenarios untestable in two dimensions (namely
three-dimensional corners) has potential to yield further information about
how the optimized operators perform numerically, as well as shed light on
optimization targets to yield better results in higher dimensions.

Furthermore, an additional unexplored topic that could prove to be a worthwhile
"fallback" goal - and one that potentially ties the two topics of this proposal
together - would be exploring the implications of the use of Kozdon and Wilcox's
projection operators on the timestep limitations of a given physical problem, and
(if a timestep disparity between the interior of a given subdomain and its interface
with another subdomain is identified) exposing this challenge to multi-rate Adams
integration.

%===========================================================================
