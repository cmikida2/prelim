\subsection{Motivation \& Background}

\subsubsection{Historical Context}

Brief literature review on the time integration, specifically
pertaining to both multi-rate integration and the state of time
integration of reacting flows, the latter of which is likely to
make specific mention of CVODE.

\subsubsection{Goals, Bounds, and Impact}

Primary goal: create a novel time integration scheme that makes
use of the benefits of multi-rate integration while also handling
the stiffness imposed by chemical kinetics in reacting flows, either
via implicit integration or adaptivity (or both), exploring the
design choices therein while verifying that this goal is met via
comparison to CVODE as a state-of-the-art alternative.

Impact: augmentation of reacting flow simulation capability in terms
of both accuracy and performance, allowing for improvement in modeling
real-world applications for combustion (subsonic/supersonic propulsion
via scramjets/ramjets, furnaces/heating, and atmospheric sciences).

%===========================================================================
\subsection{Governing Equations}

This subsection will describe in detail the Navier-Stokes equations with
chemical reactions, discussing in detail how the chemical kinetics are
incorporated via mixture-averaged mass diffusion, constant internal energy
assumptions, and NASA9 thermodynamic polynomials.

%===========================================================================
\subsection{Numerical Methods}

\subsubsection{Adams Methods}

Discussing both explicit and implicit Adams methods, and
how they are derived within Leap (similar to the JCP).

\subsubsection{Multi-rate Adams Methods}

Discussing how the Adams methods are extended into a
multi-rate integration framework (similar to the JCP).

\subsubsection{Timestep Control Algorithm}

Discussing the timestep control algorithm, its formulation,
and its potential weaknesses.

%===========================================================================
\subsection{Verification}

Discussing the test problems that will be used to verify our approach
and show its efficacy, including the demonstrative reacting mixing layer
problem as well as the Cantera two-reactor setup and the one-dimensional
laminar free flame example, while also making specific mention of metrics
by which we will gauge the success of the new time integration method(s).

%===========================================================================
\subsection{Results}

This subsection will focus on the results obtained thus far, including
recreation of the mixing layer problem as described by Blumen and Michalke,
and verification of the Leap-generated IMEX-RK scheme using PyJac via flame
speed comparison in the free flame example.

%===========================================================================
\subsection{Outlook}

\subsubsection{Current Status}

This section will include a discussion of the current state of the project,
including the fact that PyJac and Leap have been integrated into the fluid
solver and are driving the free flame example, as well as discussion of the
current state of the mixing layer problem construction (initial mode/growth
rate verification with air), and the status of the multi-rate adaptive and
explicit methods.

\subsubsection{Risk Mitigation}

This section will discuss the possibility of demonstrating improvement
over CVODE in terms of accuracy using the existing IMEX-RK scheme with
more accurate chemical Jacobians (and lack of a first-order splitting
approach) as a "fallback" goal.

%===========================================================================
