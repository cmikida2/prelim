\subsection{Motivation \& Background}

\subsubsection{Historical Context}

Brief literature review on the state of hybrid discretizations,
and what existing formulations lack/possess.

\subsubsection{Goals, Bounds, and Impact}

Primary goal: create an interface for a hybrid SBP-DG discretization
that is provably energy stable, meaning that the semi-discrete expression
for the rate of change of total energy (energy across all subblocks) can
be proven to be non-increasing. Initially, this will focus on/target the
less complex interior stencils, with the hope of extending the lessons
learned to the boundaries (is it fair to call the latter a "soft bound?").

Impact: provide a stable and accurate way to mesh complex geometries
by allowing for the simplicity of finite differences away from physical boundaries,
with the flexibility of DG near said boundaries.

%===========================================================================
\subsection{Governing Equations}

This subsection will lay out the target problem for the novel hybrid
discretization, which for now is the acoustic wave equation in two dimensions.

%===========================================================================
\subsection{Numerical Methods}

This subsection will lay out the basics of the finite difference method
(summation by parts), the basics of the DG method, and further discuss
the approach for the interface, all in subsections.

\subsubsection{Summation-by-Parts Operators}

\subsubsection{Discontinuous Galerkin Method}

\subsubsection{Interface Method}

%===========================================================================
\subsection{Validation}

This section lays out the test problem we will use to determine
whether or not our new method meets the goals laid out in the
previous section(s), and also defines the success metrics we will
use to measure this (namely, energy tracking - both to ensure
that our semi-discrete energy relation is accurate to the actual
numerical method at work, and to ensure that energy is non-increasing).

\subsection{Results}

This section will discuss the results obtained thus far, including the
verification that Kozdon and Wilcox's projection operators (our starting point)
do give an energy value that matches the semi-discrete relation used for the
proof, and a demonstration showing the subpar nature of these operators as
an ending point (this likely will be specific to the poor boundary behavior).

\subsection{Outlook}

\subsubsection{Risk Mitigation}

This section will discuss the potential fallback of extension of an existing
interface method (Kozdon and Wilcox) to three dimensions, should our goal
of a new interface method not come to fruition. Another fallback goal could
involve determining whether this existing interface method imposes a timestep
limitation and exposing this to multi-rate integration.

%===========================================================================
\subsubsection{Current Status}

This section will discuss the current status of the project, including the
hunt for a unisolvent projection operator.

%===========================================================================
